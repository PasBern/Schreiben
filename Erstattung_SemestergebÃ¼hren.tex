\documentclass[foldmarks=on,%   Faltmarken setzen
               paper=a4,%       Papierformat
	       	   10pt,%		    Schriftgr��e
               fromphone=off,%  Telefonnummer im Absender
               fromrule=off,%   keine Linie unter Absender
               fromfax=off,%    keine Faxnummer
               fromemail=on,%   Emailadresse
               fromurl=off,%    keine Homepage
	       subject=beforeopening%  Plazierung der Betreffzeile
]{scrlttr2}

\usepackage[T1]{fontenc}
\usepackage[latin1]{inputenc}
\usepackage[ngerman]{babel}
\usepackage{url}
\usepackage[scaled]{helvet}
%\usepackage{lmodern}

%%%Serifenlose Schrift f�r das gesamte Dokument
\renewcommand{\familydefault}{\sfdefault}



\begin{document}

\setkomafont{fromname}{\footnotesize}
\setkomavar{fromname}{Pascal Bernhard}

\setkomafont{fromaddress}{\footnotesize}
\setkomavar{fromaddress}{Schwalbacher Stra�e 7\\12161 Berlin}

%\setkomafont{fromphone}{\footnotesize}
%\setkomavar{fromphone}{0511 55 71 15}

%\setkomafont{fromemail}{\footnotesize}
\setkomavar{fromemail}{pascal.bernhard@belug.de}


\setkomavar{backaddressseparator}{%
~\raisebox{0.25ex}{\textbf{\Large.}}~}
\setkomafont{backaddress}{\tiny}
%%\setkomavar{backaddress}{Dietrich% k�rzere R�ckaddresse
%%\usekomavar{backaddressseparator}\usekomavar{fromaddress}}



\begin{letter}{%         Empf�nger-Adresse

Freie Universit�t Berlin\\
Iltisstra�e 1\\
14195 Berlin\\
}

\vspace{6mm}

\opening{Sehr geehrte Damen und Herren,}% Anrede

Das Bundesverfassungsgericht hat in einem Urteil vom November 2012 festgestellt, dass die gesetzliche Regelung f�r die Erhebung einer R�ckmeldegeb�hr in H�he von 100 DM bzw. sp�ter 51,13 EUR im Berliner Hochschulgesetz alter Fassung verfassungswidrig ist \textemdash{} siehe BVerfG, 2 BvL 51/06 vom 6.11.2012, online unter: %http://www.bverfg.de/entscheidungen/ls20121106_2bvl005106.html

Laut Bundesverfassungsgericht steht die H�he der Geb�hr in "`grobem Missverh�ltnis zu dem Zweck, die Kosten f�r die Bearbeitung der R�ckmeldung zu decken"' und ist daher nichtig, erst mit einer Gesetzes�nderung aus dem Jahr 2004 bestand wieder eine Rechtsgrundlage f�r die Geb�hr.

Ich beantrage daher auf Grundlage dieses Urteils die R�ckzahlung der von mir in den Jahren 2003-2005 geleisteten R�ckmeldegeb�hren und weise darauf hin, dass ich diese stets nur unter Vorbehalt gezahlt habe. Die Geb�hren wurden unter meiner damaligen Matrikelnummer 3753179 verbucht, es geht um folgende Semester:

\closing{Mit freundlichen Gr��en,}


\end{letter}

\end{document}
