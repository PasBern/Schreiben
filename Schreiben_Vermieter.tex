\documentclass[a4paper,11pt]{letter}

\usepackage[ansinew]{inputenc}
\usepackage{multicol}
\usepackage{rotating}
\usepackage{xcolor}
\usepackage[german,germanb]{babel}
\usepackage{fontenc}
\usepackage{geometry}


\usepackage[dvips]{hyperref}

\author{Pascal Bernhard}
\title{Schreiben Vorlage}
\date{\today}
%%%___________________________________________________________________________________________________________
\geometry{top=20mm}

\begin{document}
 
\pagestyle{empty}



	\begin{flushright}
		Berlin, den \today \linebreak

		Pascal Bernhard\\
		Schwalbacher Stra{\ss}e 7\\
		12161 Berlin\\		
		Telefon: \textsl{+49 176 57 13 27 90}\\
		Email: \textsl{pascal.hasko@yahoo.de}\\
		
		
	\end{flushright}

\vspace{5mm}	

Helmut H�usgen\\
Robert-Koch-Stra�e 8\\
45147 Essen

\vspace{25mm}


Sehr geehrter Herr H�usgen,\\

mir war es in den letzten zwei Wochen nicht m�glich, Sie telefonisch zu erreichen, weder unter der ``alten'' Festnetznummer (+49 20 11 77 44 23), die anscheinend nicht mehr g�ltig ist, noch unter der auf dem Briefkopf angegebenen Handy-Nummer (+49 170 478 55 23). Am Sonntag wurde ich zwar von einem Herrn H�usgen von letzterer Nummer zur�ckgerufen, jedoch erkl�rte mir der Herr, dass hier ein Mussverst�ndnis vorliege und er mit der Wohnungsvermietung nichts zu tun habe. Ehrlich gesagt, bin ich jetzt ein bisschen verwirrt.
Da eine Email vom 5. M�rz ohne Antwort geblieben ist, schreibe ich einen traditionellen Brief in der Hoffnung, Sie damit erreichen zu k�nnen. 

\textsl{Zur Erkl�rung der Mietr�ckst�nde:}\\
Im Oktober letzten Jahres wurde in Folge der Finanzkrise ein monatliches Stipendium der DIHK wegen Zahlungsschwierigkeiten noch vor dem eigentlichen Auslaufen Ende 2010 kurzfristig eingestellt. Seither konnte ich keine Arbeit finden, die mir diesen Einkommensausfall angemessen ausgleicht. Hinzukommen chronische Kosten f�r Zahnbehandlungen, welche die Krankenkasse nur zum geringen Teil, bzw. �berhaupt �bernimmt.

Meine Nachbarin Frau Volmary hat mir nun angeboten, f�r das Erste die Mietzahlung an meiner statt zu �bernehmen (sie hat mir nicht genau gesagt, welchen Betrag sie �berweisen will, daher k�nnte dies geringer als die eigentliche Miete ausfallen). Sie sollten Anfang kommender Woche eine �berweisung erhalten. 

Wie erl�utert erlaubt es mir die gegenw�rtige finanzielle Situation nicht, die angeh�uften Mietr�ckst�nde umgehend zu begleichen. Dies wird erst nach Studienabschluss (Anfang n�chsten Jahres) realistisch der Fall sein mit einer ``regul�ren`` Arbeit. Ich bitte um Ihr Verst�ndnis f�r
meine schwierige Lage. Einen anderen Vorschlag kann ich Ihnen nicht machen.


Bitte teilen Sie mir doch mit, wie ich Sie am Besten erreichen kann.

\par
%\vspace{mm}	


Mit freundlichen Gr�{\ss}en,\\ 

\emph{Pascal Bernhard}


\end{document}
